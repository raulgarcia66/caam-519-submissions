\documentclass[12 pt]{article}
\usepackage[utf8]{inputenc}
\usepackage{fullpage}
\usepackage{verbatim}
\usepackage{listings}

\title{CAAM 519, Homework \#2: \LaTeX{} Submission}

\author{\texttt{rg66}}
\date{September 30, 2021}

\begin{document}

\maketitle

\section{Communicating with remote machines via ssh}

\begin{verbatim}
1. The command to ssh into the clear machine
ssh rg66@ssh.clear.rice.edu


2. The message the clear machine prints when logging in
Warning: Permanently added the ECDSA host key for IP address '128.42.124.180' to the list of known hosts.
The Rice University Network - Unauthorized access is prohibited
rg66@ssh.clear.rice.edu's password: 
The Rice University Network
 ===========================
 Unauthorized use is prohibited.
 
 This computer system is for authorized users only.  Individuals using this
 system without authority or in excess of their authority are subject to
 having all their activities on this system monitored and recorded or
 examined by any authorized person, including law enforcement, as system
 personnel deem appropriate.  In the course of monitoring individuals
 improperly using the system or in the course of system maintenance, the
 activities of authorized users may also be monitored and recorded.  Any
 material so recorded may be disclosed as appropriate.  Anyone using this
 system consents to these terms.
 
 Problems and/or questions should be submitted via the problem tracking
 system form: http://helpdesk.rice.edu
 
CURRENT USAGE AND LOAD ON THE COMPUTE NODES:
  Thu Oct 14 16:15:01 CDT 2021

 System                   	# Users   	Load ( 5, 10, 15 minute)      
   agate.clear.rice.edu   	    5     	  0.00, 0.04, 0.05            
   amber.clear.rice.edu   	    7     	  0.03, 0.34, 0.32            
   cobalt.clear.rice.edu  	    0     	  0.05, 0.03, 0.05            
   jade.clear.rice.edu    	    2     	  0.07, 0.06, 0.22            
   onyx.clear.rice.edu    	    2     	  1.05, 1.15, 1.10            
   opal.clear.rice.edu    	    4     	  0.00, 0.04, 0.05            
   pyrite.clear.rice.edu  	    2     	  0.00, 0.02, 0.05            

NOTE: !!!!!!!!!!!!!!!!!!!!!!!!!!!!!!!!!!!!!!!!!!!!!!!!!!!!!!!!!!!!!!!!!!!
 
   Please log an RT ticket for any issues you may have.
 
   Please log a ticket at https://help.rice.edu if you have any of the following:
     Feel documentation is lacking.
     Have trouble getting into the system.
     Feel the system is missing a tool.
 
   NOTE: svn.rice.edu is now READ ONLY in preparation for removal 
         from service on January 8th, 2022
 
   NOTE: If you don't have a home drive when using Clear, please 
         create a ticket in help.rice.edu  Be sure to put 
         "need clear home directory" in the subject and one will 
         be setup for you.
 
   NOTE: New tools available.  Maven 3.8.2, gcc 9.3.1, java 11, clang 11.0.1,
         julia 1.6.1, python 3.9.1, vim 8.2 w/ python support.  
         Plus special purpose applications: 
           Circuit Design (ngspice, magic), Network Simulation (ns-3).
         See kb.rice.edu for details.
 
   CLEAR NEWS -- https://kb.rice.edu/internal/page.php?id=71856
   Tips and Hints -- https://kb.rice.edu/internal/page.php?id=71857


3. The output of > echo $HOSTNAME
pyrite.clear.rice.edu


4. The command to change the shell prompt text to > is
export PS1="> "
\end{verbatim}

\section{A script to build a latex document while hiding auxiliary files}

\begin{lstlisting}[language=bash,basicstyle=\ttfamily, caption={Contents of clean-build.sh}]
#!/bin/bash

# Check if the project name was passed
# 1st line: Checks if the number of arguments equals 0
# 2nd line: Tells the user the the format of the argument
# 3rd line: Exit the program with a 1 for not successful 
if [[ $# -eq 0 ]]; then
	echo "Need to give name of project, without extension"
	exit 1
fi

# Create the directory if it doesn't exist
# 1st line: Checks if directory does not exist
# 2nd line: Creates the directory if it doesn't
if [[ ! -d ".build" ]]; then
	mkdir .build
fi

# Compile the tex file
pdflatex $1.tex

# Keep only the .tex and .pdf files; send all else to .build
# 1st line: Loop through all files that begin with $1, followed by a period, then any extension. These files are the only ones of interest
# 2nd line: Check if the extension is not .pdf and not .tex
# 3rd line: If the above is true, then fyle is an auxiliary file we can send to .build
for fyle in $1.*; do
	if [[ $fyle != "$1.pdf" && $fyle != "$1.tex" ]]; then
		mv $fyle .build
	fi
done

\end{lstlisting}

\vspace{5mm}
\noindent Two ways the code can lead to unexpected or undesirable behavior:

1. If .build were to be created beforehand but as a file, that would lead to anerror when trying to create the .build as a directory.

2. If project.tex is passed as an argument instead of project, this would lead to an error when compiling via pdflatex. 

\end{document}
